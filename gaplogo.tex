% !TEX TS-program = lualatex
%\documentclass[class=minimal,border=0pt]{standalone}
\documentclass[class=article,border=0pt]{standalone}
\usepackage{iftex}

\ifluatex
  % instruct lualatex to generated a reproducible PDF, so e.g. no
  % dates in it, to ensure re-running TeX produces an identical PDF
  \pdfvariable suppressoptionalinfo \numexpr32+64+512\relax
\fi

% choose a color for the GAP text
\usepackage[svgnames]{xcolor}
  \definecolor{gapcolor}{named}{Crimson}
  %\definecolor{gapcolor}{named}{Teal}

\usepackage{fontspec}
  \setmainfont{Ubuntu}

\usepackage{tikz}
  \usetikzlibrary{positioning}
  \usetikzlibrary{arrows.meta}

% style for the letter G, A, P
\newcommand\gapstyle[1]{\textbf{\Huge\textcolor{gapcolor}{#1}}}

% style for the remaining letters
\ifx\ReducedMode\undefined
\newcommand\extrastyle[1]{{\LARGE{}\textcolor{black}{#1}}}
\else
\newcommand\extrastyle[1]{}
\fi

\ifx\SmallIconMode\undefined
  \def \lw {2.4pt}
\else
  % thicker lines for tiny rendering sizes
  \def \lw {2.9pt}
\fi

\begin{document}
\begin{tikzpicture}[line width=\lw, >={Stealth[inset=3pt, angle=60:10pt]}]

  \def \radius {1cm}
  \def \margin {20} % margin in angles, depends on the radius

  % set coordinates of the four circles, starting at the one on the right
  \coordinate (A) at (00:\radius) {};
  \coordinate (B) at (120:\radius) {};
  \coordinate (C) at (240:\radius) {};
  \coordinate (Z) at (0,0) {};

  % draw the circles
  \node[draw, circle] at (Z) {};
  \node[draw, circle] at (A) {};
  \node[draw, circle] at (B) {};
  \node[draw, circle] at (C) {};
  
  % draw the edges connecting the four circles / nodes
  \draw[shorten >= 2mm, shorten <=2mm] (Z) -- (A);
  \draw[shorten >= 2mm, shorten <=2mm] (Z) -- (B);
  \draw[shorten >= 2mm, shorten <=2mm] (Z) -- (C);

  % draw the three double-ended arrows which indicate swapping of the circles
  \draw[<->, color=gapcolor] ({0+\margin}:\radius) 
    arc ({0+\margin}:{120-\margin}:\radius);

  \draw[<->, color=gapcolor] ({120+\margin}:\radius) 
    arc ({120+\margin}:{240-\margin}:\radius);

  \draw[<->, color=gapcolor] ({240+\margin}:\radius) 
    arc ({240+\margin}:{360-\margin}:\radius);

\ifx\NoTextMode\undefined
    % draw the GAP text
    \node[anchor=west] at (1.2, 0.8) {\gapstyle{G}\extrastyle{roups}};
    \node[anchor=west] at (1.2, 0  ) {\gapstyle{A}\extrastyle{lgorithms}};
    \node[anchor=west] at (1.2,-0.8) {\gapstyle{P}\extrastyle{rogramming}};
\fi

\end{tikzpicture}

\end{document}
